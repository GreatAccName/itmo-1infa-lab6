\twocolumn
\topmargin=-2cm
\fontsize{9pt}{9pt}\selectfont
\pagestyle{fancy}
\fancyhead[L,C]{}
\fancyhead[R]{\footnotesize\textbf{Е. Г. Николаев}}
\fancyfoot[R]{\thepage}
\renewcommand{\headrulewidth}{0pt}
\renewcommand{\footrulewidth}{0pt}
\setcounter{page}{37}

\begin{center}
    {\bf\LARGE
    Случай с методом\\[6pt]
    математической\\[16pt]
    индукции}
\end{center}

Однажды учитель задал ученикам
10-го класса на дом следующую
задачу:

Доказать, пользуясь методом
математической индукции, что при
любом натуральном $n$ верно неравенство

\[
    P_n=\frac{1\cdot 3\cdot 5\dots
    (2n-1)}{2\cdot 4\cdot 6\dots
    2n}<\frac{1}{\sqrt{n+1}}.
\]

Коля В. нашел дома следующее
решение:

\begingroup
\renewcommand{\theenumi}{(\asbuk{enumi})}
\renewcommand{\labelenumi}{\asbuk{enumi})}
\begin{enumerate}
    \item При $n=1$ неравенство верно,
    так как
    \[
        P_1=\frac{1}{2}<\frac{1}{\sqrt{2}}
    \]
    \item Покажем, что из условия
    $P_n<\frac{1}{\sqrt{n+1}}$ вытекает
    справедливость неравенства
    $P_{n+1}<\frac{1}{\sqrt{n+2}}$.
\end{enumerate}
\endgroup

Для этого д\,о\,с\,т\,а\,т\,о\,ч\,н\,о показать,
что

\[
    \frac{P_{n+1}}{P_n}
    <\frac{\sqrt{n+1}}{\sqrt{n+2}},
\]
или
\[
    \frac{2n+1}{2n+2}<\sqrt{\frac{n+1}{n+2}},
\]
или
\begin{align}
    4n^3+12n^2+9n+2< \notag\\
    <4n^3+12n^2+12n+4, \notag
\end{align}
или
\[
    0<3n+2,
\]
что, очевидно, верно. Таким образом,
справедливость доказываемого
неравенства следует из приципа
математической индукции.

Однако, рассказывая в школе это
решение, Коля забыл поставить в
знаменателе единицу и стал доказывать
более <<грубое>> неравенство:
\[
    P_n<\frac{1}{\sqrt{n}}.
\]
Вот что у него получилось:

\begingroup
\renewcommand{\theenumi}{(\asbuk{enumi})}
\renewcommand{\labelenumi}{\asbuk{enumi})}
\begin{enumerate}
    \item При $n=1$ неравенство верно,
    так как
    \[
        P_1=\frac{1}{2}<\frac{1}{1}.
    \]
    \item Покажем, что из условия
    $P_n<\frac{1}{\sqrt{n}}$ следует
    $P_{n+1}<\frac{1}{\sqrt{n+1}}$.
\end{enumerate}
\endgroup

Для этого достаточно показать, что

\[
    \frac{P_{n+1}}{P_n}=
    \frac{\sqrt{n}}{\sqrt{n+1}},
\]
или
\[
    \frac{2n+1}{2n+2}=\sqrt{\frac{n}{n+1}},
\]
или
\[
    (2n+1)^2(n+1)<4(n+1)^2n,
\]
или
\[
    4n^3+8n^2+5n+1<4n^3+8n^3+4n,
\]
или
\[
    n+1<0.
\]

Получился абсурд. Более точное
неравенство оказалось и более
легким для доказательства. Почему?

Разобравшимся в этом вопросе
предлагаем доказать два неравенства:

\begin{enumerate}
    \item $P_n<\frac{1}{\sqrt{3,1n}}.$\\
    Интересно, что в этом неравенстве
    коэффициент 3,1 нельзя
    заменить на 3,2. Самый большой
    коэффициент, при котором оно остатся
    верным, равен $\pi$ (отношение длины
    окружности к диаметру).
    \item $
        \frac{1}{2^3}+\frac{1}{3^3}+\cdots
        +\frac{1}{n^3}<\frac{1}{4}.
    $
\end{enumerate}