\twocolumn
\fontsize{9pt}{9pt}\selectfont
\pagestyle{fancy}
\fancyhf{}
\fancyfoot[R]{\thepage}
\renewcommand{\headrulewidth}{0pt}
\renewcommand{\footrulewidth}{0pt}
\setcounter{page}{59}

{\bf 5.} Определить угол треугольника,
в котором медиана, биссектриса и
высота делят угол на четыре равные
части.

\begin{center}
    {\textcolor{red}{\bf\large Разбор задач первого тура}}
\end{center}

{\bf 1.} Ясно, что каждое из восьми
срезаемых у куба по углам тел является
правильной треугольной пирамидой
с прямыми углами между
боковыми ребрами (рис. 1). Если

\begin{figure}[h]
    \noindent\centering{
        \includegraphics[width=0.7\linewidth]{../pict/1.png}
        \label{p1}
    }\\
    \noindent\centering{Рис. 1.}
    %caption{}
\end{figure}

обозначить сторону основания пирамиды
через b, то
$AD=\frac{b\sqrt{3}}{2}$,
$OD=\frac{b\sqrt{3}}{6}$,
$CD=\frac{b}{2}$,
и тогда по теореме
Пифагора высота пирамиды
$EO=\frac{b\sqrt{6}}{6}$.
Площадь основания равна
$\frac{b^2\sqrt{3}}{4}$.

\begin{figure}[b]
    \noindent\centering{
        \includegraphics[height=28mm,width=36mm]{../pict/2a.png}
    }\\
    \noindent\centering{Рис. 2а.}
\end{figure}

Поэтому объем такой пирамиды равен
$\frac{b^3\sqrt{2}}{24}$,
а объем тела,
получающегося из куба срезом весьми
таких пирамд, равен
\begin{equation}
    \label{eq:V1}
    a^3-\frac{b^3\sqrt{2}}{3}.
\end{equation}

Так как по условию задачи от
граней куба должны остаться
правильные многоугольники, то
решений два. Могут быть восьмиугольники
или квадраты (рис. 2 {\it а}, {\it б}).

В первом случае
$b=a(\sqrt{2}-1)$,
и тогда из~\eqref{eq:V1} после преобразований
получаем
$V=\frac{7a^3(\sqrt{2}-1)}{3}$.

Во втором случае
$b=\frac{a\sqrt{2}}{2}$,
и тогда
из~\eqref{eq:V1} находим
$V=\frac{5a^3}{6}$.

Заметим, что лишь 20\% школьников
нашли оба решения задачи.
Остальные находили только одно
решение. При этом предпочтения не
было для кого-либо одного из двух
многогранников --- каждый получил
приблизительное равное число голосов.

\begin{figure}[b]
    \noindent\centering{
        \includegraphics[height=28mm,width=36mm]{../pict/2b.png}
    }\\
    \noindent\centering{Рис. 2б.}
\end{figure}

\begin{center}
    \tiny
    \begin{tabular}{|c|c|c|c|c|c|c|}    \hline
        \cellcolor{yellow!50}Фигура &
        \cellcolor{yellow!50}\begin{sideways} Ко\-роль \end{sideways} &
        \cellcolor{yellow!50}\begin{sideways} Ферзь \end{sideways} &
        \cellcolor{yellow!50}\begin{sideways} Ладья \end{sideways} &
        \cellcolor{yellow!50}\begin{sideways} Слон \end{sideways} &
        \cellcolor{yellow!50}\begin{sideways} Конь \end{sideways} &
        \cellcolor{yellow!50}\begin{sideways} Пеш\-шка \end{sideways} \\ \hline
        \cellcolor{green!50}Число & & & & & & \\
        \cellcolor{green!50}внешней & 9 & 5 & 8 & 8 & 12 & 32 \\  
        \cellcolor{green!50}устойчивости & & & & & & \\ \hline
        \cellcolor{green!50}Число & & & & & & \\
        \cellcolor{green!50}внутренней & 16 & 8 & 8 & 14 & 32 & 32 \\
        \cellcolor{green!50}устойчивости & & & & & & \\ \hline
    \end{tabular}
    \noindent\centering{\small Табл. 1.}\\
\end{center}

По причине требований к <<лабе>>
пришлось добавить таблицу\footnote[1]{Таблица со страницы 51.}
и сноску.
